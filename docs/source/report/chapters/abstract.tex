\chapter{Abstract}
\label{Abstract}
This document provides a comprehensive overview of the implementation of  of a DLX microprocessor, a project undertaken by Group 02 as part of the Microelectronic Systems course at Politecnico di Torino. The project aims to enhance understanding of microprocessor design stack: starting from a RTL VHDL description to the Physical Design process. The report details the structure and the process that lead to the implementation. 
To allow the reader to understand the design choices, the report is structured as follows:
\begin{itemize}
    \item \textbf{Chapter 1: Introduction} - Introduces the DLX microprocessor architecture, its instruction set, and the project objectives.
    \item \textbf{Chapter 2: DLX Structure} - Discusses the design process, including the control unit, Datapath, and functional units.
    \item \textbf{Chapter 3: Synthesis} - Describes the testing methodologies used to validate the design and ensure correctness.
    \item \textbf{Chapter 4: Physical Design} - Covers the synthesis process and the physical design steps taken to prepare the microprocessor for fabrication.
    \item \textbf{Chapter 5: Tools and Scripts} - Lists the tools and technologies used throughout the project, including VHDL, ModelSim/QuestaSim, Innovus, Docker and custom scripts.
    \item \textbf{Chapter 6: Conclusions} - Summarizes the project outcomes, challenges faced, and lessons learned.
    \item \textbf{Appendices} - Provides additional resources, including VHDL code, user scripts, and TCL templates used in the project.
\end{itemize}  
Prior knowledge of VHDL and digital design principles is assumed, as the report focuses on the implementation details and design choices made during the project. Also a general knowledge of computer architectures is recommended, mostly in RISC principles pipelined architectures.
The document serves as a valuable resource for understanding the complexities of microprocessor design and the practical application of theoretical concepts in a real-world project.

\chapter{Conclusions}

The DLX microprocessor project has been a significant undertaking that has allowed us to apply theoretical knowledge in a practical context. The design and implementation of the DLX architecture have provided a comprehensive understanding of microprocessor design, from the Control Unit and Datapath to the synthesis and physical design processes.
The project has highlighted the importance of a well-structured design process, including the use of VHDL for hardware description, simulation for verification, and synthesis for implementation. The challenges faced during the project, such as ensuring correct timing and functionality, have reinforced the need for meticulous testing and validation at each stage of the design.
The use of custom scripts, such as Marley, Pupi, Briciola, Scorzetta, Boh, and Cassiopeia, has improved the development process, allowing efficient management of the design workflow. These tools have facilitated the compilation, simulation, synthesis, and physical design tasks, demonstrating the value of automation in hardware design.

\section{Further Work}
Future work could focus on enhancing the DLX microprocessor design by exploring advanced features such as out-of-order execution, dynamic advanced branch prediction, and cache memory implementation. Additionally, further optimization of the synthesis and physical design processes could lead to improved performance and reduced power consumption, like seen in the Synthesis and Optimization course, we could apply 
a multi-Vth algorithm to swap high leakage cells with lower, trading a decrease of power consumption with slower cell timing.
Changing the encoding of the instructions to a more compact format could also be considered, which would reduce the number of control words needed, thus reducing the overall memory.
Regarding the scripts, we could improve the user experience by adding more features, such as integration with other design tools. Furthermore, to enhance the readability of waveforms generated during simulation we could use scripts like vcd2wavedrom, which could facilitate better analysis and debugging of the design, and improve the quality of the documentation.
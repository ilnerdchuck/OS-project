\chapter{Introduction}

\section{Project Objectives}

The goal of this project is to create a simulation environment for the NXP S32K3X8EVB board using QEMU. This environment will allow developers to test and develop embedded software applications without the need for physical hardware, thereby reducing costs and increasing accessibility. The specific objectives of the project include:
\begin{itemize}
    \item Setting up QEMU to emulate the NXP S32K3X8EVB board.
    \item Configuring necessary peripherals and interfaces, such as UART and SPI, to ensure accurate simulation of the board's functionality.
    \item Porting FreeRTOS to the simulated environment to enable real-time operating system capabilities.
    \item Testing the simulation environment with sample applications to validate the implementation.
    \item Documenting the implementation process, challenges encountered, and solutions devised to overcome them.
\end{itemize}

In particular the pheripherals that will be implemented are:
\begin{itemize}
    \item UART (Universal Asynchronous Receiver/Transmitter)
    \item SPI (Serial Peripheral Interface)
\end{itemize}

\section{Tools and Technologies}

\subsection{NXP S32K3X8EVB Board}

The NXP S32K3X8EVB board is a development platform designed for automotive and industrial applications. It is based on the S32K3 series of microcontrollers, which feature an ARM Cortex-M7 core. The board includes various peripherals, such as UART, SPI, I2C, ADC, and GPIOs, allowing developers to interface with external devices and sensors. The S32K3X8EVB board is widely used in the automotive industry for applications such as motor control, body electronics, and sensor fusion.

\subsection{QEMU}

QEMU (Quick Emulator) is an open-source machine emulator and virtualizer that allows users to run operating systems and applications for one machine on a different machine. It supports a wide range of architectures, including ARM, x86, MIPS, and PowerPC, making it a versatile tool for simulating various hardware platforms.
QEMU operates in two primary modes: full system emulation and user-mode emulation. In full system emulation, QEMU emulates an entire hardware platform, including the CPU, memory, and peripherals, allowing users to run complete operating systems. In user-mode emulation, QEMU can run applications compiled for one architecture on another architecture by translating system calls and library functions.
QEMU's flexibility and extensibility make it an ideal choice for simulating embedded systems. It provides a rich set of features, including support for various peripherals, networking capabilities, and the ability to create custom device models. Additionally, QEMU's open-source nature allows developers to modify and extend its functionality to suit their specific needs.

\subsection{FreeRTOS}
FreeRTOS is an open-source real-time operating system (RTOS) designed for embedded systems. It provides a lightweight and efficient kernel that enables multitasking, inter-task communication, and resource management. FreeRTOS is widely used in various applications, including automotive, industrial, and consumer electronics, due to its simplicity, portability, and scalability.
It offers a rich set of features, such as task scheduling, semaphores, queues, and timers, allowing developers to create complex real-time applications with ease. FreeRTOS is also highly configurable, enabling users to tailor the kernel to their specific requirements by selecting only the necessary components.

\subsection{Development Environment}
The development environment for this project includes a combination of software tools and libraries to facilitate the implementation and testing of the simulation environment. Key components of the development environment include:
\begin{itemize}
    \item \textbf{QEMU}: The primary tool for emulating the NXP S32K3X8EVB board. The vesion used in this project is QEMU 9.0.0.
    \item \textbf{NXP Toolchain}: A collection of compilers and tools for building embedded applications, including the ARM GCC compiler for compiling code for the ARM Cortex-M7 architecture provided by NXP.
    \item \textbf{FreeRTOS Source Code}: The source code for FreeRTOS, which will be ported to the simulated environment.
    \item \textbf{Debugging Tools}: Tools such as GDB (GNU Debugger) for debugging and testing the simulation environment.
    \item \textbf{Version Control System}: Git is used for managing the source code and tracking changes throughout the development process.
\end{itemize}

\section{Project Structure}
The project is organized into several key components, each responsible for a specific aspect of the simulation environment. The main components include:


\begin{forest}
    for tree={
        font=\ttfamily,
        grow'=0,
        child anchor=west,
        parent anchor=south,
        anchor=west,
        calign=first,
        inner xsep=7pt,
        edge path={
          \noexpand\path [draw, \forestoption{edge}]
          (!u.south west) +(7.5pt,0) |- (.child anchor) pic {folder} \forestoption{edge label};
        },
        % style file node 
        file/.style={edge path={\noexpand\path [draw, \forestoption{edge}]
          (!u.south west) +(7.5pt,0) |- (.child anchor) \forestoption{edge label};},
          inner xsep=2pt,font=\small\ttfamily
                     },
        before typesetting nodes={
          if n=1
            {insert before={[,phantom]}}
            {}
        },
        fit=band,
        before computing xy={l=15pt},
      }  
  [group10
    [qemu
      [hw
        [arm
            [S32K3X8EVB.c - Board implementation file, file]
            [S32K3X8EVB\_MCU.c - MCU implementation file, file]
        ]
        [char
            [S32K3X8EVB\_UART.c - UART implementation file, file]
        ]
        [ssi
            [S32K3X8EVB\_SPI.c - SPI implementation file, file]
        ]
      ]
      [include
        [arm
            [S32K3X8EVB.h - Board header file, file]
            [S32K3X8EVB\_MCU.h - MCU header file, file]
        ]
        [char
            [S32K3X8EVB\_UART.h - UART header file, file]
        ]
        [ssi
            [S32K3X8EVB\_SPI.h - SPI header file, file]
        ]
      ]
    ]
    [FreeRTOS\_App - FreeRTOS application
      [main.c - Main application file, file]
      [FreeRTOSConfig.h - FreeRTOS configuration file, file]
      [pheripherals.c - Function for the pheripherals, file]
      [pheripherals.h - Header file for the pheripherals, file]
      [Makefile - Build file, file]
    ]
    [gcc-10.2.0-Earmv7GCC-eabi - NXP toolchain
    ]
    [material
      [S32K3.pdf - MCU reference manual, file]
      [S32K3X8EVB-Q289HWUM.pdf - Board user manual, file]
      [S32K3xx\_memory\_map.xls - Memory and Pheripheral map, file]
    ]
    [Docs
      [Source
        [report - Source files for the report]
        [presentation - Source files for the presentation]
        [Group10\_Presentation.pdf, file]
        [Group10\_Report.pdf, file]
      ]
    ]
    [qemu\_old - QEMU version 9.0.42 (not complaint to our implementation)]
  ]
\end{forest}

